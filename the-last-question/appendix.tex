\thispagestyle{empty}
{
\parindent0mm
\pagestyle{empty}
\raggedbottom
\newpage

\textbf{Background}\\

%From time to time I find articles, texts or poems which I really like. But
%in some of those cases the text is typeset in such a poor way that it makes
%me sad. 

It is a wonderful and unique experience to read an article, text or poem
which in some way thrills you. It may touch you in some way, it may inspire
or motivate you or it may make you cover the ending with your hand to
prevent your eyes from catching a glimpse of the ending.

But sometimes such a wonderful experience is dulled when the text is typeset 
in a horrible way. When the font doesn't fit the text, the lines are too
long or as much text as possible is squeezed onto a page.

For me, typography has the same role as rhetoric---the way in which
you say something can have a totally different impact. In my opinion this 
is equally true for the way in which typography presents a text:
a document which is typeset in a fitting manner has the ability to affect 
the reader in a totally different way.

\begin{center}
	``\textit{Typography---the voice of the page.}''
\end{center}

\pagebreak

\ \\ \ \\
This document is the result of a series of works in which I started to improve
the horrible typesetting of wonderful texts.
%This document is the result of such an effort.

I chose to typset this specific story after I had read it and was thrilled
by it. But sadly the typesetting of the document was horrible. 
%It was set using an unfitting font, with improper settings. All computer 
% speech was
%set in uppercase, being hard for the eye to read.
However, it held a lot of potential: typesetting the computer speech or the different 
time zones for examples. 
I also liked the idea of using a grotesque font as a body text font, in
order to support the futuristic manner of the story.
%When I arrived at the last page I took one hand and coverd the ending sentences
%in order for my eyes not to accidentally catch a word. You know a story
%is well written once it makes you do that.
I experimented with various layouts and fonts and at one point even had
different, increasingly futuristic, fonts for the different time chapters.
In the end, however, I decided to go for the less obtrusive approach.

\pagebreak
\ \\
\pagebreak
\ \\



\pagebreak
}

{
\parindent0mm
%\pagestyle{empty}
\raggedbottom

\fancyfoot{}
\fancyfoot[C]{
	\centering
	Michael M\"uller\\
	\pc{http://micha.elmueller.net}
}

\textbf{Typograhy Info}\\ \ \\

This text was typeset using the \textsc{pf din text condensed}. 
The computer output was set using \pc{OCR-A}, a font which was designed
as a mean to ease the transformation of printed text into a digital 
representation. This is achieved by scanning texts and using optical 
text recognition algorithms on the scanned characters.\\

For the ornaments I used the wonderful \textsc{ptl roletta floral ornaments}
font. The initials were typeset using two different fonts:  
the \textsc{p22 arts and crafts} and the \textsc{final frontier} font used
in Star Trek.
The titlepage was set using the wonderful \textsc{gotham}---a masculine,
nearly monospaced, font designed by Tobias Frere-Jones for the GQ magazine.\\

On the software side this document was typeset using \engine.
The sourcecode used to render this document is accessible via
\pc{https://github.com/cmichi/\allowbreak{}reworking-typography}. 
%A pdf version of this document is available there as well.
%\hspace{-0.1cm} \pc{https://github.com/cmichi/\-reworking-typography}\hspace{-0.1cm}. 

%\ \\
%\textbf{Inspired?}\\

%If this document motivated you to learn more about typesetting I would
%suggest to either read books about typograhy (

%watch talks by font-face designers like Spiekermann or 

%read about names like Spiekermann, Frutiger, Zapf, Hoefler or
%Frere-Jones. I would also suggest to start using high-quality typesetting
%systems. 

%\enginetwo{} is free software, you can freely download and install 
%it right now: \pc{https://www.tug.org/texlive/}.
%Maybe you got some inspirations for typesetiing, maybe you were motivated
%to learn something about Typography or high-quality typesetting systems
%like \LaTeX.
\pagebreak


%\ \\
\textbf{What to do with this document now?}\\

%What should you do with this document now? 
Of course you are free to do whatever you want with it, but I would suggest 
you either keep it, pass it on or leave it at some place where others will 
likely find it and read it.\\

%Please don't throw it away. 

%\vspace{9.1cm}
%\hspace{-3cm}
%\begin{minipage}[b]{4cm}
	%\centering
	%Michael M\"uller\\
	%\pc{http://micha.elmueller.net}
%\end{minipage}


\clearpage
\newpage
\ \\
\newpage
\thispagestyle{empty}
}
